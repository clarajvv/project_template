\section{Discusión}

Una de las principales apreciaciones de los resultados es que, en ambos caso, el tipo de acción predomintante son los \textbf{inhibidores}, que disminuyen o cesan la acción o creación de otra sustancia.

Por otro lado, podemos ver los mecanimos de acción más presentes:
\begin{enumerate}
\item \textbf{Inhibidor de DNA polimerasa}, fundamentales para la replicación del ADN.
\item \textbf{Inhibidor 26s proteosoma}, un complejo protéico que se encarga de degradar proteínas no necesarias o dañadas.
\item \textbf{Inhibidor de la bomba de Sodio/Potasio}, una enzima que realiza varias funciones en la celula, como mantener el gradiente Sodio/Potasio, el potencial de membrana o de transporte. 
\item \textbf{Inhibidor E3 ubiquitina ligasa}, proteína que recluta una enzima de ubiquitina E2, reconoce un sustrato de proteína y ayuda o cataliza directamente la transferencia de ubiquitina desde el E2 al sustrato de proteína.
\end{enumerate}

Tiene sentido que muchos de nuestros fármacos compartan mecanismo de acción ya que están destinados a grupos de proteínas relacionadas, que comparten función biológica. 

\newpage
