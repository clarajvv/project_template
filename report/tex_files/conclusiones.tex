\section{Conclusiones}

El objetivo de esta investigación ha sido encontrar fármacos que puedan combatir el virus SARS-Cov-2 usando la base de datos ChEMBL. Para ello, hemos comenzado nuestro trabajo buscando el interactoma funcional SARS-humano en la literatura científica. Aunque se han realizado numerosos artículos en poco tiempo, aún no se ha descubierto el interactoma funcional completo. Debido a ello, decidimos ampliar la búsqueda de fármacos usando proteínas humanas que interaccionan directamente con las proteínas humanas del interactoma funcional.

Aunque intentamos partir de un conjunto numeroso de proteínas de segundo grado, no ha sido posible debido a que se pierden muchas de ellas al mapear su código de gen al código ChEMBL. Esto se debe a que no existen fármacos en ChEMBL cuyo objetivo sean estas proteínas. Aún así, gracias a las proteínas de segundo grado se consiguen otros ocho fármacos potenciales de fase cuatro. 

Dentro de los fármacos obtenidos, encontramos algunos mecanismos de acción que son más frecuentes que los demás. Como se ha dicho previamente, esto puede deberse a que algunas de las proteínas comparten función biológica. También, pueden tratarse de mecanismos muy estudiados. 

Para futuros trabajos, sería interesante utilizar la información química obtenida de los fármacos potenciales para estudiar el efecto de todos ellos sobre la acción del virus, así como las rutas que siguen en el organismo humano. Además, otra línea de investigación sería estudiar cuál de las proteínas del interactoma funcional es una candidata óptima a desarrollar un fármaco, simulando la repercusión que tendría en el sistema. 

\newpage
