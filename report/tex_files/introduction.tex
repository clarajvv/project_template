\section{Introducción}
%\cite{Shulla2011}
A día de hoy, puede que la enfermedad más mencionada en todo el mundo sea el SARS-CoV-2 (coronavirus 2 del síndrome respiratorio agudo severo). Este virus, que se originó en la ciudad China de Wuhan en diciembre de 2019, se ha extendido a todos los países del mundo rápidamente. Aunque la tasa de mortalidad del 2\% no es especialmente alta, el SARS-CoV-2 es altamente contagioso y transmisible de persona en persona con un período de incubación de hasta 24 días \cite{Yan2020}. Este brote de SARS-CoV2 ha provocado grades impactos en la salud social y la economía en todos los niveles, convirtiéndo en una prioridad encontrar tratamientos eficientes para la enfermedad. 

Como podemos ver en la revisión bibliográfica \cite{Yan2020}, ya en los primeros 75 días del brote de COVID-19 se propusieron algunos tratamientos médicos para combatir el SARS-CoV-2. Los más usados fueron antivirales (oseltamivir, ganciclovir y KALETRA), agentes antibióticos (cefalosporina, quinolona y carbapenem) para prevenir infecciones secundarias, y en los casos más graves terapia de oxígeno y terapia de reemplazo renal. El uso de estos tratamientos se debe en su mayoría a la eficacia que han tenido en virus similares como es la influenza. 

Para maximizar la eficacia de los tratamientos empleados, es vital diseñar fármacos que ataquen al virus de forma directa. Sin embargo, esto supone dos inconvenientes. El primero es conocer una forma eficaz de combatir la enfermedad y el segundo es el largo proceso que conlleva diseñar un fármaco. 

Encontramos numerosos artículos que tratan de desentrañar el interactoma SARS-humano (\cite{Hoffmann2020}, \cite{Yan2020a}, \cite{Shulla2011}, \cite{Chan2020}, \cite{Gysi2020}). Conociendo las proteínas humanas que son afectadas por el virus, podemos encontrar fármacos que perjudiquen la acción de éste en los seres humanos. 

Debido a la urgencia de encontrar tratamientos efectivos, en nuestra investigación se propone encontrar fármacos ya existentes que actúen sobre las proteínas del interactoma entre SARS y humanos. 

Para hacer esta búsqueda, usaremos la base de datos ChEMBL. ChEMBL es un conjunto de datos 'quimogenómicos' que une información sintética, de bioactividad y genómica de medicamentos. Surgió ante la necesidad de indexar productos genéticos y conectarlos con medicamentos para su posterior análisis y divulgación por parte de farmacéuticas, pequeñas organizaciones y establecimientos académicos. 

En nuestro caso, ChEMBL nos permitirá obtener fármacos cuyas dianas sean proteínas humanas del interactoma SARS-Humano.

\newpage